\chapter*{Acrónimos}
\label{ch::acro}

% #   ATENÇÃO
% A lista de acrónimods deve ser ordenada alfanumericamente.
% Estrangeirismos devem ser realçados em itálico.
% Se o relatório for escrito em Inglês, uma palavra portuguesa é um estrangeirismo.

% O maior acrónimo deve ser colocado neste ponto (reparar que CFIUTE é maior que TCP!).
%               vvvvvv
\begin{acronym}[SIGGRAPH]
    \acro{API}{\emph{Application Programming Interface}}
    \acro{ASCII}{\emph{American Standard Code for Information Interchange}}
    \acro{CBO}{\emph{Color Buffer Object}}
    \acro{CG}{Computação Gráfica}
    \acro{CPU}{\emph{Central Processing Unit}}
    \acro{CUDA}{\emph{Compute Unified Device Architecture}}
    \acro{FOV}{\emph{Field of View}}
    \acro{fps}{\emph{frames per second}}
    \acro{GLAD}{\emph{Multi-Language GL/GLES/EGL/GLX/WGL Loader-Generator}}
    \acro{GLEW}{\emph{OpenGL Extension Wrangler Library}}
    \acro{GLM}{\emph{OpenGL Mathematics}}
    \acro{GLSL}{\emph{OpenGL Shader Language}}
    \acro{GPU}{\emph{Graphics Processing Unit}}
    \acro{GUI}{\emph{Graphical User Interface}}
    \acro{IDE}{\emph{Integrated Development Environment}}
    \acro{MVP}{\emph{Model-View-Projection}}
    \acro{RAM}{\emph{Random Access Memory}}
    \acro{RM}{Ressonância Magnética}
    \acro{SDF}{\emph{Signed Distance Function}}
    \acro{SGI}{\emph{Silicon Graphics}}
    \acro{SIGGRAPH}{\emph{Special Interest Group on Computer Graphics and Interactive Techniques}}
    \acro{SISM}{\emph{Smart Interactive State Machine}}
    \acro{SoC}{\emph{System on a Chip}}
    \acro{SSBO}{\emph{Shader Storage Buffer Object}}
    \acro{SWOT}{\emph{Strength, Weakness, Opportunity, and Threat Analysis}}
    \acro{TI}{Tecnologias de Informação}
    \acro{UBI}{Universidade da Beira Interior}
    \acro{UC}{Unidade Curricular}
    \acro{URI}{\emph{Uniform Resource Identifier}}
    \acro{VAO}{\emph{Vertex Attribute Object}}
    \acro{VBO}{\emph{Vertex Buffer Object}}
    \acro{VRAM}{\emph{Video Random-Access Memory}}
    \acro{WSL}{\emph{Windows Subsystem for Linux}}
    \acro{WSLg}{\acl{WSL} \emph{\acs{GUI}}}
\end{acronym}
