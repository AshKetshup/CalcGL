\chapter{Tecnologias e Ferramentas}
\label{ch::tecno}

\section{Introdução}
\label{sec::tecno:intro}

Bla bla inicial\ldots


\section{Tecnologias}
\label{sec::tecno:tecno}

Para a implementação do projeto \theapp~com \opengl~foi escolhida a linguagem de programação C++. A fim de melhorar a experiência de utilização do \opengl, foram utilizadas as seguintes bibliotecas e \textit{frameworks}:

\begin{itemize}
    %\item \textbf{C++}: Linguagem compilada de uso geral multi-paradigma. Desenvolvida por Bjarne Stroustrup.
    
    \item \textbf{GLFW~\cite{glfw}}: \ac{API} simplificada para o \opengl, igualmente multi-plataforma, permitindo a gestão de janelas, contextos, superfícies e comandos (rato, teclado e \textit{joystick});
    
    \item \textbf{\acs{GLAD}~\cite{glad,glad-webservice}} (\acl{GLAD}): gerador automático de \textit{loaders} para \opengl;
    
    \item \textbf{\acs{GLM}~\cite{glm}} (\acl{GLM}): biblioteca matemática baseada na linguagem dos \textit{shaders} do \opengl, \ac{GLSL}.
    
    \item \textbf{\textit{FreeType}~\cite{freetype}}: biblioteca de desenvolvimento dedicada à renderização de fontes em \textit{bitmaps} utilizáveis, por exemplo, pelo \opengl.
    
    % \item \textbf{\acs{GLSL}} (\acl{GLSL}): principal linguagem de \textit{shading} para \opengl.
    
    % \item \textbf{\opengl~\cite{opengl}}: \ac{API} multi-plataforma e com suporte a múltiplas linguagens de programação para a renderização de gráficos vetoriais 2D e 3D com recurso à placa gráfica;
\end{itemize}

Há ainda a notar que a \ac{GPU} é programada através de \textit{shaders} com a linguagem \acf{GLSL}.

As versões destas bibliotecas e outro \textit{software} auxiliar utilizado estão sumariados na Tabela \ref{tab::ferramentas}.

\begin{table}[!p]
    \centering
    \caption[Ferramentas utilizadas]{Ferramentas e tecnologias utilizadas, organizadas por categoria.}
    \label{tab::ferramentas}
    \begin{tabular}{p{1cm} l l}
        \toprule
        & {\bfseries \textit{Software} / Tecnologia} & {\bfseries Versão} \\
        \midrule
        \multicolumn{3}{l}{\bfseries Aplicação \opengl} \\
        & \opengl           & 4.6 \\
        & GLFW              & 3.3.5 \\
        & \acs{GLAD}        & 0.1.34 \\
        & \acs{GLM}         & 0.9.9.8 \\
        & \textit{FreeType} & 2.10.4 \\
        \midrule
        \multicolumn{3}{l}{\bfseries Relatório} \\
        & Xe\TeX & 3.141592653-2.6-0.999993 \\
        &\textit{TeXstudio}\ccopyright & 4.2.2 \\
        \midrule
        \multicolumn{3}{l}{\bfseries Controlo de versões} \\
        & \textit{git} & 2.36.1 \\
        & \textit{GitKraken} & 8.6.1  \\
        \bottomrule
    \end{tabular}
\end{table}



\section{Código \emph{Open Source}}
\label{sec::tecno:opensource}

Além das bibliotecas e \textit{frameworks} referidas na Secção \ref{sec::tecno:tecno}, código \textit{open source} adicional foi utilizado para facilitar a implementação de componentes que não fazem parte do objetivo de estudo do projeto. Estes são:

\begin{itemize}
    \item \textbf{\textit{CParser}} \todo{Citação}: biblioteca para \textit{parsing} de uma sequência de caracteres como uma expressão usando o algoritmo \textit{Shunting-yard} de Dijkstra;
    
    \item \todo{Código do Learn OpenGL}
    
    \item \hint{quickGL??}
\end{itemize}


\section{\textit{Hardware}}
\label{sec::tecno:hw}

O \textit{software} final foi testado em três computadores distintos, listados na Tabela \ref{tab::hardware}.

\begin{table}[!p]
	\centering
	\caption[Lista de \textit{hardware} para testes]{Lista de \textit{hardware} onde o projeto \theapp~foi testado.}
	\label{tab::hardware}
	\begin{tabular}{p{1cm} l l}
		\toprule
		%& {\bfseries Componente} & {\bfseries } \\
		%\midrule
		\multicolumn{3}{l}{\bfseries Computador portátil 1} \\
		& Processador (\acs{CPU})   & Intel\registered~i5-8300H (2.3--4.0GHz) \\
		& Placa gráfica (\acs{GPU}) & NVidia\registered~GTX 1050 Ti (4GB) \\
		& Memória \acs{RAM}         & 16GB (DDR4 2133MHz) \\
		& Armazenamento             & 256GB (PCIe 3.0 x4) \\
		\midrule
		\multicolumn{3}{l}{\bfseries Computador portátil 2} \\
		& Processador (\acs{CPU})   & AMD Ryzen\texttrademark~9 5900HS (3.0--4.6GHz) \\
		& Placa gráfica (\acs{GPU}) & NVidia\registered~RTX 3050 Ti (4GB) \\
		& Memória \acs{RAM}         & 16GB (DDR4 3200MHz) \\
		& Armazenamento             & SSD NVMe 1TB (PCIe 3.0 x4) \\
		\midrule
		\multicolumn{3}{l}{\bfseries Computador \textit{desktop}} \\
		& Processador (\acs{CPU})   & AMD Ryzen\texttrademark~7 2700X (3.7--4.3GHz) \\
		& Placa gráfica (\acs{GPU}) & NVidia\registered~RTX 3070 (8GB) \\
		& Memória \acs{RAM}         & 32GB (DDR4 3200MHz) \\
		& Armazenamento             & SSD NVMe 1TB (PCIe 3.0 x4) \\
		\bottomrule
	\end{tabular}
\end{table}


\section{Conclusões}
\label{sec::tecno:conc}

\ldots Whiskas Saquetas.
