\chapter{Tecnologias e Ferramentas}
\label{ch::tecno}

\section{Introdução}
\label{sec::tecno:intro}

Bla bla inicial\ldots


\section{Tecnologias}
\label{sec::tecno:tecno}

\begin{itemize}
    \item \textbf{C++}: Linguagem compilada de uso geral multi-paradigma. Desenvolvida por Bjarne Stroustrup.
    
    \item \textbf{\textit{FreeType}~\cite{freetype}}: biblioteca de desenvolvimento dedicada à renderização de fontes em \textit{bitmaps} utilizáveis, por exemplo, pelo \opengl.
    
    \item \textbf{GLFW~\cite{glfw}}: \ac{API} simplificada para o \opengl, igualmente multi-plataforma, permitindo a gestão de janelas, contextos, superfícies e comandos (rato, teclado e \textit{joystick});
    
    \item \textbf{\ac{GLAD}~\cite{glad,glad-webservice}}: gerador automático de \textit{loaders} para \opengl;
    
    \item \textbf{\ac{GLM}~\cite{glm}}: biblioteca matemática baseada na linguagem dos \textit{shaders} do \opengl, \ac{GLSL}.
    
    \item \textbf{\ac{GLSL}}: principal linguagem de \textit{shading} para \opengl.
    
    \item \textbf{\opengl~\cite{opengl}}: \ac{API} multi-plataforma e com suporte a múltiplas linguagens de programação para a renderização de gráficos vetoriais 2D e 3D com recurso à placa gráfica;
\end{itemize}

\begin{table}[!htbp]
    \centering
    \begin{tabular}{p{1cm} l l}
        \toprule
        & {\bfseries \textit{Software} / Tecnologia} & {\bfseries Versão} \\
        \midrule
        \multicolumn{3}{l}{\bfseries Aplicação \opengl} \\
        & \opengl           & 4.6 \\
        & GLFW              & 3.3.5 \\
        & \acs{GLAD}        & 0.1.34 \\
        & \acs{GLM}         & 0.9.9.8 \\
        & \textit{FreeType} & 2.10.4 \\
        \midrule
        \multicolumn{3}{l}{\bfseries Relatório} \\
        & Xe\TeX & 3.141592653-2.6-0.999993 \\
        &\textit{TeXstudio}\textsuperscript{\textcopyright} & 4.2.2 \\
        \midrule
        \multicolumn{3}{l}{\bfseries Controlo de versões} \\
        & \textit{git} & 2.36.1 \\
        & \textit{GitKraken} & 8.6.1  \\
        \bottomrule
    \end{tabular}
    \caption[Ferramentas utilizadas]{Ferramentas e tecnologias utilizadas, organizadas por categoria.}
    \label{tab::ferramentas}
\end{table}



\section{Código \emph{Open Source}}
\label{sec::tecno:opensource}

% Shaders de exemplo, cparse\ldots
\begin{itemize}
    \item \textbf{\textit{CParser}}: biblioteca para \textit{parsing} de uma sequência de caracteres como uma expressão usando o algoritmo \textit{Dijkstra's Shunting-yard};
    
     
\end{itemize}

\section{Conclusões}
\label{sec::tecno:conc}

\ldots Whiskas Saquetas.
