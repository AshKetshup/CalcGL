\chapter{Conclusões e Trabalho Futuro}
\label{ch::conc}

\section{Conclusões}
\label{sec::conc:conc}

Com o desenvolvimento e teste deste projeto, foi possível:
\begin{itemize}
    \item Estudar as funções implícitas e o método de cálculo das respetivas iso-superfícies;
    \item Analisar o modelo de renderização por volume;
    \item Estudar o algoritmo da área de \ac{CG} para renderização de superfícies implícitas por \textit{ray marching} assim como \textit{sphere tracing};
    \item Aprofundar o conhecimento na linguagem \ac{GLSL};
    \item Conhecer em maior profundidade o funcionamento do \opengl~moderno;
\end{itemize}

Com o conjunto destes, foi então possível alcançar o objetivo principal: desenvolver e implementar um sistema de visualização de funções implícitas com recurso ao algoritmo de \textit{ray marching} e \opengl~moderno.

Em relação ao motor de cálculo por software, a utilização do mesmo revela não ser de nenhuma forma apropriada para qualquer uso, tendo em conta os longos tempos de espera necessários para calculo de uma única \textit{frame}.

O algoritmo de \textit{sphere tracing}, ainda sendo uma alternativa com uma melhoria de performance considerável, é revelada atualmente inalcançável no contexto de funções implícitas. Isto, devido à inexistência de um método geral para cálculo dinâmico de \ac{SDF} relativa a uma função implícita.


\section{Trabalho Futuro}
\label{sec::conc:futuro}

Ainda que, no contexto do projeto proposto, tenham sido cumpridos os objetivos definidos, a conclusão deste revelou a existência de variadas oportunidades de melhoria e estudo nas áreas envolvidas.

É possível o recurso a outros algoritmos, assim como o uso de \textit{compute shaders} para melhorar a performance do programa.

Seria também interessante explorar a área matemática intrínseca ao cálculo de uma \ac{SDF}, com o propósito de alcançar um método geral de computação para qualquer função implícita.

Por seu turno, a testagem do código para diferentes configurações de \textit{hardware}, com o intuito de correlacionar a performance obtida com a configuração do sistema.