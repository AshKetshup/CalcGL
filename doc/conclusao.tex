\chapter{Conclusões e Trabalho Futuro}
\label{ch::conc}

\section{Conclusões}
\label{sec::conc:conc}

Com o desenvolvimento e teste deste projeto foi possível:
\begin{itemize}
    \item Estudar as funções implícitas e o método de cálculo das respetivas iso-superfícies;
    \item Analisar o modelo de renderização por volume;
    \item Estudar o algoritmo da área de \ac{CG} para renderização de superfícies implícitas por \textit{ray marching} assim como \textit{sphere tracing};
    \item Aprofundar o conhecimento na linguagem \ac{GLSL};
    \item Conhecer em maior profundidade o funcionamento do \opengl~moderno.
\end{itemize}

Desta forma foi alcançado o objetivo principal: desenvolver e implementar um sistema de visualização de funções implícitas com recurso ao algoritmo de \textit{ray marching} pelo \opengl~moderno.

Em relação ao motor de cálculo por \textit{software}, a utilização do mesmo revela não ser adequado para uso prático tendo em conta os longos tempos de execução necessários para cálculo de uma única \textit{frame}.

O algoritmo naïve de \textit{ray marching} revelou-se adequado para a renderização de qualquer função implícita com excepção dos fractais. Estes necessitam de um nível de precisão que excede as capacidades do \textit{hardware} testado. Seria necessário conhecer a \acs{SDF} específica de cada fractal a ser renderizado a fim de se poder usar \textit{sphere tracing} e, assim, renderizá-los.

Por seu turno, o algoritmo de \textit{sphere tracing}, apesar de apresentar melhorias de \textit{performance} consideráveis face ao algoritmo naïve de \textit{ray marching}, tem um uso mais limitado no contexto de funções implícitas devido à inexistência de um método generalizado para cálculo da \acf{SDF} de uma superfície implícita arbitrária. É, contudo, o algoritmo preferencial para superfícies implícitas bem conhecidas.


\section{Trabalho Futuro}
\label{sec::conc:futuro}

Ainda que, no contexto do projeto proposto, tenham sido cumpridos os objetivos definidos, a conclusão deste revelou a existência de oportunidades de melhoria e estudo nas áreas envolvidas, nomeadamente o recurso a outros algoritmos, assim como o uso de \textit{compute shaders} para melhorar a \textit{performance} do programa. Por outro lado, a optimização do algoritmo naïve poderá ter um impacto positivo na renderização de fractais com este algoritmo.

Seria igualmente relevante explorar a área matemática intrínseca ao cálculo de uma \ac{SDF}, com o propósito de alcançar um método generalizado para qualquer função implícita.
