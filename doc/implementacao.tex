\chapter{Implementação}
\label{ch::impl}

%\section{Introdução}
%\label{sec::impl:intro}

A implementação do projeto \theapp~será detalhada no presente Capítulo, com particular foco nos seguintes pontos:

\begin{itemize}[nosep]
	\item Requisitos funcionais do projeto;
	\item Estrutura do código e o fluxo do mesmo;
	\item Detalhes de implementação e desafios encontrados.
\end{itemize}


\section{Funcionalidades e Requisitos}
\label{sec::impl:requisitos}

O projeto \theapp~segue uma lista de \textbf{funcionalidades} a implementar baseados nos objetivos do projeto (Secção \ref{sec::intro:objetivos}), em específico:

\begin{itemize}
    \item Renderização de funções implícitas com uso de \textit{ray marching};
    \item Suporte a dois motores de cálculo:
    \begin{enumerate}[nosep]
        \item Por \textit{software} (i.e. os cálculos são efetuados pela \ac{CPU});
        \item Aceleração por \textit{hardware} (com uso da \ac{GPU});
    \end{enumerate}
    \item Suporte a ficheiros externos contendo as funções implícitas a renderizar.
\end{itemize}

Foram ainda considerados os seguintes \textbf{requisitos} adicionais:
\begin{itemize}
    \item Abrir qualquer ficheiro \verb*|.function| dinamicamente por um meio gráfico (\ac{GUI});
    \item Capacidade de personalizar a cor da superfície implícita;
    \item Apresentar as \ac{fps} alcançadas pelo programa para análise dos resultados.
\end{itemize}


\section{Lógica e Estruturação}
\label{sec::impl:estrutura}

\hint{Estrutura do código e respetivo fluxo.}




\section{Detalhes de Implementação}
\label{sec::impl:detalhes}

\hint{Explicar como as funções são injetadas diretamente no fragment shader.}

\hint{Aceleração por \textit{hardware}.}

\hint{GPU foi abaixo.}






%\section{Conclusões}
%\label{sec::impl:conc}
