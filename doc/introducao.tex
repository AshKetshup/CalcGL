\chapter{Introdução}
\label{ch::intro}

\section{Enquadramento}
\label{sec::intro:enquadramento}

O estudo matemático de funções levou à descoberta de diferentes métodos para a sua representação e análise, nomeadamente os modelos explícitos, paramétricos e implícitos. Contudo, apenas as funções explícitas são trivialmente renderizadas por computadores; estas representam grandes desafios uma vez que nem sempre é possível obter expressões explícitas.

Neste sentido, o estudo de funções implícitas torna-se vital para inúmeros fins, pelo que a sua renderização computacional tem sido alvo de estudo por décadas. Os resultados dos enormes avanços feitos na área são atualmente desfrutados por milhões de pessoas, tanto a título pessoal como no mundo empresarial e de investigação.


\section{Motivação}
\label{sec::intro:motivacao}

As versões modernas do \opengl~(i.e. a partir da versão 3.3) permitem que o programador escreva os seus próprios \textit{shaders} a serem utilizados na \textit{pipeline} de renderização. Uma consequência imediata é a possibilidade de paralelizar inúmeros cálculos que normalmente seriam realizados pela \ac{CPU}.

Ainda assim, uma miríade de \textit{software} de renderização não tira proveito de tais capacidades. Desta forma, é de interesse estudar um algoritmo de renderização por volume e analisar a sua potencial paralelização.


\section{Objetivos}
\label{sec::intro:objetivos}

O presente projeto tem por \textbf{objetivo principal} implementar um sistema de visualização de funções implícitas.

Este tem ainda os seguintes \textbf{objetivos secundários}:
\begin{itemize}
	\item Estudar funções implícitas e o cálculo das respetivas iso-superfícies;
	\item Analisar o modelo de renderização por volume;
	\item Implementar o algoritmo de \textit{ray marching} em particular;
	\item Implementar aceleração por \textit{hardware} através da programação de \emph{shaders} em \opengl.
\end{itemize}


\section{Organização do Documento}
\label{sec::intro:orgdoc}

O presente relatório estrutura-se em seis capítulos:

\begin{enumerate}
	\item No primeiro capítulo --- \textbf{Introdução} --- é apresentado o projeto, em particular o seu enquadramento e motivação, assim como os seus objetivos e a respetiva organização do relatório.
	
	\item No segundo capítulo --- \textbf{Estado da Arte} \revision{} --- são apresentados os conceitos fundamentais de funções implícitas, assim como métodos usados para renderizar focando principalmente no algoritmo \textit{Ray Marching}. É também feita uma introdução aos conceitos introdutórios do \opengl moderno.
	
	\item No terceito capítulo --- \textbf{Tecnologias e Ferramentas} \revision{} --- são definido as tecnologias usadas no desenvolvimento do projeto assim como as ferramentas.
	
	\item No quarto capítulo --- \textbf{Implementação} \revision{} --- são descritas as decisões e detalhes tomados durante o desenvolvimento, em conjunto com o funcionamento do programa e os seus requisitos.
	
	\item No quinto capítulo --- \textbf{Testes e Resultados} \revision{} --- é demonstrado resultados do processo de teste ao programa final.
	
	\item No sexto capítulo --- \textbf{Conclusões e Trabalho Futuro} \revision{} --- é refletido sobre os conhecimentos obtidos ao longo do desenvolvimento do projeto, os seus obstáculos, o que não foi possível e que poderá ser no futuro explorado.
\end{enumerate}
