\chapter{Estado da Arte}
\label{ch::arte}

\section{Introdução}
\label{sec::arte:intro}

A visualização computacional de funções é a base de muitas aplicações práticas em computação gráfica, tais como em videojogos e visualização molecular. O estudo destas funções permite igualmente outro tipo de cálculos, tais como colisões.

Para alcançar os objetivos propostos no presente projeto, é imperativo estudar os seguintes tópicos:

\begin{itemize}
	\item Funções implícitas;
	\item Técnicas de renderização em geral e algoritmos volumétricos em particular;
	\item Uso da \ac{API} \opengl e programação em \ac{GLSL}.
\end{itemize}


\section{Funções Implícitas}
\label{sec::arte:implicitas}

\subsection{Definição e Aplicações}
\label{ssec::arte:implicitas:def}

O que é? Exemplo(s). Que aplicações têm?

\subsection{Desafios Computacionais}
\label{ssec::arte:implicitas:desafios}

Renderização em computação gráfica. Que métodos existem? Que alternativas estão em aberto?


\section{Técnicas de Renderização}
\label{sec::arte:render}

Breve introdução às categorias de técnicas/algoritmos de renderização.

\subsection{Renderização por Volume}
\label{ssec::arte:render:volume}

O que é ``volume rendering''? Que exemplos de algoritmos existem?

\subsection{\emph{Ray Marching}}
\label{ssec::arte:render:raymarch}

Como funciona o algoritmo? É paralelizável? Se sim, como e porquê?


\section{\opengl}
\label{sec::arte:opengl}

Não recomendo um rip-off do meu relatório, mas ele pode servir de base para esta secção, tentando melhorá-lo e corrigir possíveis gafes.


\section{Conclusões}
\label{sec::arte:conc}

\ldots Whiskas Saquetas.
