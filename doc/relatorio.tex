\documentclass[12pt,a4paper]{memoir}

% substituir linha seguinte por 
% \usepackage[english]{babel} 
% se o relatório for escrito na língua inglesa.
\usepackage[portuguese]{babel}

% \usepackage[utf8]{inputenc}
\usepackage[T1]{fontenc}

\usepackage{makeidx}
\usepackage{xspace}
\usepackage{graphicx,color,times}
\usepackage{fancyhdr}
% \usepackage{pxfonts}
% \usepackage{times}
% \usepackage{mathptm}
% \usepackage{amssymb}
% \usepackage{amsfonts}

\usepackage{minted}
\usepackage{paralist}
\usepackage{enumitem}
\usepackage{booktabs}
\usepackage{amssymb}
\usepackage{calc}
\usepackage{array}
%\usepackage{color}
%\usepackage{colortbl}
\usepackage{xcolor}

\usepackage{caption}
\usepackage{subcaption}

\usepackage{amsmath}
\usepackage{latexsym}
\usepackage[printonlyused]{acronym}
\usepackage{float}
\usepackage{listings}
\usepackage{tocbibind}
% \usepackage{natbib}
\usepackage{hyperref}

% \usepackage{glossaries}
% \makeglossaries

% \renewcommand{\ttdefault}{phv}

\pagestyle{fancy}
\renewcommand{\chaptermark}[1]{\markboth{#1}{}}
\renewcommand{\sectionmark}[1]{\markright{\thesection\ #1}}
\fancyhf{} \fancyhead[LE,RO]{\bfseries\thepage}
\fancyhead[LO]{\bfseries\rightmark}
\fancyhead[RE]{\bfseries\leftmark}
\renewcommand{\headrulewidth}{0.5pt}
\renewcommand{\footrulewidth}{0pt}
\setlength{\headheight}{15.6pt}
\setlength{\marginparsep}{0cm}
\setlength{\marginparwidth}{0cm}
\setlength{\marginparpush}{0cm}
\addtolength{\hoffset}{-1.0cm}
\addtolength{\oddsidemargin}{\evensidemargin}
\addtolength{\oddsidemargin}{0.5cm}
\addtolength{\evensidemargin}{-0.5cm}


\usepackage{fix-cm}
\usepackage{fourier}
\usepackage[scaled=.92]{helvet}
\definecolor{ChapGrey}{rgb}{0.6,0.6,0.6}
\newcommand{\LargeFont}{
	\usefont{\encodingdefault}{\rmdefault}{b}{n}
	\fontsize{60}{80}\selectfont\color{ChapGrey}
}
\makeatletter
\makechapterstyle{GreyNum}{
	\renewcommand{\chapnamefont}{\large\sffamily\bfseries\itshape}
	\renewcommand{\chapnumfont}{\LargeFont}
	\renewcommand{\chaptitlefont}{\Huge\sffamily\bfseries\itshape}
	\setlength{\beforechapskip}{0pt}
	\setlength{\midchapskip}{40pt}
	\setlength{\afterchapskip}{60pt}
	\renewcommand\chapterheadstart{\vspace*{\beforechapskip}}
	\renewcommand\printchaptername{
		\begin{tabular}{@{}c@{}}
			\chapnamefont \@chapapp\\}
		\renewcommand\chapternamenum{\noalign{\vskip 2ex}}
		\renewcommand\printchapternum{\chapnumfont\thechapter\par}
		\renewcommand\afterchapternum{
		\end{tabular}
		\par\nobreak\vskip\midchapskip}
	\renewcommand\printchapternonum{}
	\renewcommand\printchaptertitle[1]{
		{\chaptitlefont{##1}\par}}
	\renewcommand\afterchaptertitle{\par\nobreak\vskip \afterchapskip}
}
\makeatother
\chapterstyle{GreyNum}

\setcounter{tocdepth}{3}
\setsecnumdepth{subsubsection}

\renewcommand{\ttdefault}{lmtt}


% NEW COLORS
\definecolor{dark}{gray}{0.25}
\definecolor{lgray}{gray}{0.9}
\definecolor{dkblue}{rgb}{0,0.13,0.4}
\definecolor{dkgreen}{rgb}{0,0.6,0}
\definecolor{gray}{rgb}{0.5,0.5,0.5}
\definecolor{mauve}{rgb}{0.58,0,0.82}

\lstset{ %
	language=C,                    basicstyle=\footnotesize,
	numbers=none,                  numberstyle=\tiny\color{gray}, 
	stepnumber=1,                  numbersep=5pt,
	backgroundcolor=\color{white}, showspaces=false,
	showstringspaces=false,        showtabs=false,
	frame=single,                  rulecolor=\color{black},
	tabsize=2,                     captionpos=b,
	breaklines=true,               breakatwhitespace=false,
	title=\lstname,                keywordstyle=\color{blue},
	commentstyle=\color{dkgreen},  stringstyle=\color{mauve},
	escapeinside={\%*}{*)},        morekeywords={*},
	belowskip=0cm
}

% \renewcommand{\lstlistingname}{Excerto de Código}
% \renewcommand{\lstlistlistingname}{Lista de Excertos de Código}

% \renewcommand{\today}{\day \ifcase \month \or janeiro\or fevereiro\or março\or %
	% abril\or maio\or junho\or julho\or agosto\or setembro\or outubro\or novembro\or %
	% dezembro\fi de \number \year}

\newcommand{\famousquote}[2]{
	\begin{quote}
		\rule{\textwidth-2\leftmargin}{0.4pt}
		{\itshape #1}
		\vspace{-12pt}
		\begin{flushright}
			\textasciitilde~#2
		\end{flushright}
		\vspace{-20pt}
		\rule{\textwidth-2\leftmargin}{0.4pt}
	\end{quote}
}

\graphicspath{{./img/}}
\newcommand{\usecasescale}{0.5}

\newcommand{\theteam}{Diogo Simões}
\newcommand{\theapp}{\emph{CalcGL}}
\newcommand{\theappdescription}{Visualização de funções implícitas por \emph{ray marching}}
\newcommand{\groupname}{{\itshape\theteam}}
\newcommand{\appname}{\emph{\theapp}:~{\emph{\theappdescription}}}
\newcommand{\opengl}{\textit{OpenGL}\textsuperscript{\textregistered}}

\newcommand{\revision}[1]{{\color{red}[Rev] #1}}
\newcommand{\todo}[1]{{\color{red}[TODO] #1}}
\newcommand{\hint}[1]{{\color{green!50!black}[Hint] #1}}


\begin{document}
	\thispagestyle{empty}
\setcounter{page}{-1}

\begin{center}
\begin{Huge}
\textbf{Universidade da Beira Interior}
\end{Huge}
\end{center}

\begin{center}
\begin{Huge}
Departamento de Informática
\end{Huge}
\end{center}

\vspace{0.07cm}
\begin{figure}[!htb]
\centering
\includegraphics[width=191pt]{ubi-fe-di.png}
\end{figure}

\vspace{\vspacepre}
\begin{center}
\begin{Large}
\textbf{N\textordmasculine{} 121 --- 2022}\\
\textbf{\emph{\theappdescription}}
\end{Large}
\end{center}


\vspace{\vspacepre}
\begin{center}
\begin{normalsize}
\begin{large}
Elaborado por:
\end{large}
\end{normalsize}
\end{center}

\vspace{\vspaceinner}
\begin{center}
\begin{large}
\textbf{Diogo Castanheira Simões}
\end{large}
\end{center}

\vspace{\vspacepre}
\begin{center}
\begin{normalsize}
\begin{large}
Orientador:
\end{large}
\end{normalsize}
\end{center}

\vspace{\vspaceinner}
\begin{center}
\begin{large}
\textbf{Professor Doutor Abel João Padrão Gomes}
\end{large}
\end{center}

%\vspace{\vspacepre}
%\begin{center}
%\begin{normalsize}
%\begin{large}
%Co-orientador:
%\end{large}
%\end{normalsize}
%\end{center}
%
%\vspace{\vspaceinner}
%\begin{center}
%\begin{large}
%\textbf{Mestre Igor Cordeiro Bordalo Nunes}
%\end{large}
%\end{center}



\vspace{\vspacepre}
\begin{center}
\begin{normalsize}
\today
\end{normalsize}
\end{center}

	
	\clearpage{\thispagestyle{empty}\cleardoublepage}
	\frontmatter
	\chapter*{Agradecimentos}
\label{ch::ack}

Em primeiríssimo lugar, agradeço à minha família que independentemente da situação sempre se demonstraram dispostos a me apoiar durante todo o meu percurso académico. Em destaque aos meus pais, pois foram eles que permitiram que tenha chegado onde atualmente estou.

Agradeço ao meu professor orientador, Prof. Doutor Abel João Padrão Gomes por ter aceitado o desafio em me orientar e pela confiança depositada em mim durante esta jornada.

Aos meus dois melhores amigos, Raquel Guerra e Igor Nunes, estou eternamente grato pelo apoio incondicional dado desde o dia em que nos conhecemos. Sem as suas amizades certamente não seria a pessoa que sou hoje.

Ao Igor Nunes, quero ainda demonstrar a minha gratidão pelo suporte dado durante o desenvolvimento deste projeto, atuando como um \textit{quasi} coorientador.

Ao meu amigo Pedro Cavaleiro pelas noitadas de boa disposição e companhia assim como as refrescantes subidas à Serra.

Não esquecendo da companhia e camaradagem dos meus amigos Cristiano Santos e Pedro Batista.

Em último, mas não menos importante, um agradecimento aos meus restante amigos que, mesmo sem serem aqui mencionados por nome, estiveram sempre disponíveis para um café e conversa.
	
	\clearpage{\thispagestyle{empty}\cleardoublepage}
	\tableofcontents
	
	\clearpage{\thispagestyle{empty}\cleardoublepage}
	\listoffigures
	
	% #   ATENÇÃO
	% Se não existirem tabelas, comentar as duas linhas seguintes
	\clearpage{\thispagestyle{empty}\cleardoublepage}
	\listoftables
	
	% #   ATENÇÃO
	% Se existirem trechos de código, descomentar as seguintes linhas
	% \clearpage{\thispagestyle{empty}\cleardoublepage}
	% \lstlistoflistings
	
	\clearpage{\thispagestyle{empty}\cleardoublepage}
	\chapter*{Acrónimos}
\label{ch::acro}

% #   ATENÇÃO
% A lista de acrónimods deve ser ordenada alfanumericamente.
% Estrangeirismos devem ser realçados em itálico.
% Se o relatório for escrito em Inglês, uma palavra portuguesa é um estrangeirismo.

% O maior acrónimo deve ser colocado neste ponto (reparar que CFIUTE é maior que TCP!).
%               vvvvvv
\begin{acronym}[SIGGRAPH]
    \acro{API}{\emph{Application Programming Interface}}
    \acro{ASCII}{\emph{American Standard Code for Information Interchange}}
    \acro{CBO}{\emph{Color Buffer Object}}
    \acro{CG}{Computação Gráfica}
    \acro{CPU}{\emph{Central Processing Unit}}
    \acro{CUDA}{\emph{Compute Unified Device Architecture}}
    \acro{GLAD}{\emph{Multi-Language GL/GLES/EGL/GLX/WGL Loader-Generator}}
    \acro{GLEW}{\emph{OpenGL Extension Wrangler Library}}
    \acro{GLM}{\emph{OpenGL Mathematics}}
    \acro{GLSL}{\emph{OpenGL Shader Language}}
    \acro{GPU}{\emph{Graphics Processing Unit}}
    \acro{GUI}{\emph{Graphical User Interface}}
    \acro{mRNA}{\ac{RNA} mensageiro}
    \acro{MVP}{\emph{Model-View-Projection}}
    \acro{PDB}{\emph{Protein Data Bank}}
    \acro{pre-mRNA}{\ac{RNA} pré-mensageiro}
    \acro{RAM}{\emph{Random Access Memory}}
    \acro{RM}{Ressonância Magnética}
    \acro{RNA}{Ácido Ribonucleico}
    \acro{SGI}{\emph{Silicon Graphics}}
    \acro{SIGGRAPH}{\emph{Special Interest Group on Computer Graphics and Interactive Techniques}}
    \acro{SISM}{\emph{Smart Interactive State Machine}}
    \acro{SMART}{\emph{\acs{SMILES} arbitrary target specification}}
    \acro{SMILES}{\emph{Simplified molecular-input line-entry system}}
    \acro{SoC}{\emph{System on a Chip}}
    \acro{SSBO}{\emph{Shader Storage Buffer Object}}
    \acro{SWOT}{\emph{Strength, Weakness, Opportunity, and Threat Analysis}}
    \acro{TAC}{Tomografia Axial Computorizada}
    \acro{TI}{Tecnologias de Informação}
    \acro{UBI}{Universidade da Beira Interior}
    \acro{UC}{Unidade Curricular}
    \acro{URI}{\emph{Uniform Resource Identifier}}
    \acro{VAO}{\emph{Vertex Attribute Object}}
    \acro{VBO}{\emph{Vertex Buffer Object}}
    \acro{VDW}{\emph{van der Waals}}
    \acro{WSL}{\emph{Windows Subsystem for Linux}}
    \acro{WSLg}{\acl{WSL} \emph{\acs{GUI}}}
\end{acronym}

	
	% \clearpage{\pagestyle{empty}\cleardoublepage}
	% \include{glossario}
	
	\clearpage{\thispagestyle{empty}\cleardoublepage}
	
	\mainmatter
	\acresetall
	\chapter{Introdução}
\label{ch::intro}

\section{Enquadramento}
\label{sec::intro:enquadramento}


\section{Motivação}
\label{sec::intro:motivacao}


\section{Objetivos}
\label{sec::intro:objetivos}


\section{Organização do Documento}
\label{sec::intro:orgdoc}

O presente relatório estrutura-se em seis capítulos:

\begin{enumerate}
	\item No primeiro capítulo --- \textbf{Introdução} --- é apresentado o projeto, em particular o seu enquadramento e motivação, assim como os seus objetivos e a respetiva organização do relatório.
	
	\item No segundo capítulo --- \textbf{Estado da Arte} --- .
	
	\item No terceito capítulo --- \textbf{Tecnologias e Ferramentas} --- .
	
	\item No quarto capítulo --- \textbf{Implementação} --- .
	
	\item No quinto capítulo --- \textbf{Testes} --- .
	
	\item No sexto capítulo --- \textbf{Conclusões e Trabalho Futuro} --- .
\end{enumerate}

	\clearpage{\thispagestyle{empty}\cleardoublepage}
	\chapter{Estado da Arte}
\label{ch::arte}

\section{Introdução}
\label{sec::arte:intro}

A visualização computacional de funções é a base de muitas aplicações práticas em computação gráfica, tais como em videojogos e visualização molecular. O estudo destas funções permite igualmente outro tipo de cálculos, tais como colisões.

Para alcançar os objetivos propostos no presente projeto, é imperativo estudar os seguintes tópicos:

\begin{itemize}
	\item Funções implícitas;
	\item Técnicas de renderização em geral e algoritmos volumétricos em particular;
	\item Uso da \ac{API} \opengl e programação em \ac{GLSL}.
\end{itemize}


\section{Funções Implícitas}
\label{sec::arte:implicitas}

\subsection{Definição e Aplicações}
\label{ssec::arte:implicitas:def}

% O que é? Exemplo(s). Que aplicações têm?

Às funções definidas em função de uma variável dá-se o nome de \textbf{funções explícitas}. Exemplos clássicos em $\mathbb{R}^2$ incluem equações de retas ($y = mx + b$) e parábolas ($y = ax^2 + bx + c$). Ora, nem todos os subconjuntos de pontos no espaço cartesiano podem ser definidos por funções explícitas. Um exemplo comum em $\mathbb{R}^2$ é a circunferência:

\begin{equation}
	(x - x_0)^2 + (y - y_0)^2 = r^2
	\label{eq::circ_implicita}
\end{equation}

onde $(x_0, y_0)$ é o centro e $r$ o raio.

Sendo uma equação de segundo grau, é possível representá-la através de duas funções explícitas:

\begin{eqnarray}
		y = y_0 + \sqrt{r^2 - (x - x_0)^2} \\
		y = y_0 - \sqrt{r^2 - (x - x_0)^2}
\end{eqnarray}

Esta transformação só é possível até polinómios de grau 4, tornando-se impossível para graus superiores. Contudo, estas expressões polinomiais continuam a ser subconjuntos válidos de $\mathbb{R}^n$, necessitando então de formas alternativas de representação. Dois métodos e respetivas representações da circunferência são:

\begin{enumerate}
	\item \textbf{Funções paramétricas}: cada eixo é definido em ordem a uma variável adicional $t$:
	\begin{equation}
		\left\{\begin{array}{l}
			x = r\cos(t) \\
			y = r\sin(t)
		\end{array}\right.
	\label{eq::circ_parametrica}
	\end{equation}
	
	\item \textbf{Funções implícitas}: a equação não é definida a ordem a uma variável em particular (equação (\ref{eq::circ_implicita})).
\end{enumerate}

Uma função implícita é então definida por $f~:~\mathbb{R}^n \longrightarrow \mathbb{R}$, ou seja, para qualquer ponto em $\mathbb{R}^n$ é determinado um resultado em $\mathbb{R}$. Dependendo da função, o valor obtido pode ter significado, tal como uma grandeza física (\textit{e.g.} densidade de um líquido ou sua temperatura a cada ponto do espaço). Esta função diz-se \textbf{algébrica} caso seja polinomial em cada variável.

Por seu turno, em $\mathbb{R}^3$, a \textbf{iso-superfície} de uma função implícita é a superfície que satisfaz a condição $f(\mathbf{x}) = 0$ (onde, doravante, $\mathbf{x} \equiv (x,y,z)$). Esta pode ser suavizada através de um parâmetro $s \in \mathbb{R}$ tal que $f(\mathbf{x}) - s = 0$.


\todo{Mais aplicações.}


\subsection{Desafios Computacionais}
\label{ssec::arte:implicitas:desafios}

\hint{Renderização em computação gráfica. Que métodos existem? Que alternativas estão em aberto?}


\section{Técnicas de Renderização}
\label{sec::arte:render}

\hint{Breve introdução às categorias de técnicas/algoritmos de renderização.}

\subsection{Renderização por Volume}
\label{ssec::arte:render:volume}

\hint{O que é ``volume rendering''? Que exemplos de algoritmos existem?}

\subsection{\emph{Ray Marching}}
\label{ssec::arte:render:raymarch}

\hint{Como funciona o algoritmo? É paralelizável? Se sim, como e porquê?}


\section{\opengl}
\label{sec::arte:opengl}

\hint{Não recomendo um rip-off do meu relatório, mas ele pode servir de base para esta secção, tentando melhorá-lo e corrigir possíveis gafes.}


\section{Conclusões}
\label{sec::arte:conc}

\ldots Whiskas Saquetas.

	\clearpage{\thispagestyle{empty}\cleardoublepage}
	\chapter{Tecnologias e Ferramentas}
\label{ch::tecno}

\section{Introdução}
\label{sec::tecno:intro}

Bla bla inicial\ldots


\section{Tecnologias}
\label{sec::tecno:tecno}

\begin{itemize}
    \item \textbf{C++}: Linguagem compilada de uso geral multi-paradigma. Desenvolvida por Bjarne Stroustrup.
    
    \item \textbf{\textit{FreeType}~\cite{freetype}}: biblioteca de desenvolvimento dedicada à renderização de fontes em \textit{bitmaps} utilizáveis, por exemplo, pelo \opengl.
    
    \item \textbf{GLFW~\cite{glfw}}: \ac{API} simplificada para o \opengl, igualmente multi-plataforma, permitindo a gestão de janelas, contextos, superfícies e comandos (rato, teclado e \textit{joystick});
    
    \item \textbf{\ac{GLAD}~\cite{glad,glad-webservice}}: gerador automático de \textit{loaders} para \opengl;
    
    \item \textbf{\ac{GLM}~\cite{glm}}: biblioteca matemática baseada na linguagem dos \textit{shaders} do \opengl, \ac{GLSL}.
    
    \item \textbf{\ac{GLSL}}: principal linguagem de \textit{shading} para \opengl.
    
    \item \textbf{\opengl~\cite{opengl}}: \ac{API} multi-plataforma e com suporte a múltiplas linguagens de programação para a renderização de gráficos vetoriais 2D e 3D com recurso à placa gráfica;
\end{itemize}

\begin{table}[!htbp]
    \centering
    \begin{tabular}{p{1cm} l l}
        \toprule
        & {\bfseries \textit{Software} / Tecnologia} & {\bfseries Versão} \\
        \midrule
        \multicolumn{3}{l}{\bfseries Aplicação \opengl} \\
        & \opengl           & 4.6 \\
        & GLFW              & 3.3.5 \\
        & \acs{GLAD}        & 0.1.34 \\
        & \acs{GLM}         & 0.9.9.8 \\
        & \textit{FreeType} & 2.10.4 \\
        \midrule
        \multicolumn{3}{l}{\bfseries Relatório} \\
        & Xe\TeX & 3.141592653-2.6-0.999993 \\
        &\textit{TeXstudio}\textsuperscript{\textcopyright} & 4.2.2 \\
        \midrule
        \multicolumn{3}{l}{\bfseries Controlo de versões} \\
        & \textit{git} & 2.36.1 \\
        & \textit{GitKraken} & 8.6.1  \\
        \bottomrule
    \end{tabular}
    \caption[Ferramentas utilizadas]{Ferramentas e tecnologias utilizadas, organizadas por categoria.}
    \label{tab::ferramentas}
\end{table}



\section{Código \emph{Open Source}}
\label{sec::tecno:opensource}

% Shaders de exemplo, cparse\ldots
\begin{itemize}
    \item \textbf{\textit{CParser}}: biblioteca para \textit{parsing} de uma sequência de caracteres como uma expressão usando o algoritmo \textit{Dijkstra's Shunting-yard};
    
     
\end{itemize}

\section{Conclusões}
\label{sec::tecno:conc}

\ldots Whiskas Saquetas.

	\clearpage{\thispagestyle{empty}\cleardoublepage}
	\chapter{Implementação}
\label{ch::impl}

%\section{Introdução}
%\label{sec::impl:intro}

A implementação do projeto \theapp~será detalhada no presente Capítulo, com particular foco nos seguintes pontos:

\begin{itemize}[nosep]
	\item Requisitos funcionais do projeto;
	\item Estrutura do código e o fluxo do mesmo;
	\item Detalhes de implementação;
\end{itemize}


\section{Funcionalidades e Requisitos}
\label{sec::impl:requisitos}

\hint{Uma breve lista de requisitos funcionais.}\\
Para que o desenvolvimento do projeto seja o mais suave possível, foram descritas as \textbf{funcionalidades} a implementar tendo em conta os objetivos do projeto \todo{citar objetivos}, mais em especifico:

\begin{itemize}
    \item Renderização de funções implícitas com uso de \textit{Ray Marching};
    \item Suporte a dois motores de cálculo:
    \begin{enumerate}
        \item Por software (os cálculos são efetuados pela \ac{CPU});
        \item Acelerado por \textit{hardware} (por \ac{GPU});
    \end{enumerate}
    \item Suporte a ficheiros externos contendo as funções implícitas;
\end{itemize}

Foram ainda considerados os seguintes \textbf{requisitos}:
\begin{itemize}
    \item Abrir qualquer ficheiro \verb*|.function| dinamicamente por um meio gráfico (\ac{GUI});
    \item Capacidade de personalizar a cor da superfície implícita resultante;
    % \item Visualizar os \ac{FPS};
\end{itemize}


\section{Lógica e Estruturação}
\label{sec::impl:estrutura}

\hint{Estrutura do código e respetivo fluxo.}


\section{Detalhes de Implementação}
\label{sec::impl:detalhes}

\hint{Explicar como as funções são injetadas diretamente no fragment shader.}

\hint{Aceleração por \textit{hardware}.}



%\section{Conclusões}
%\label{sec::impl:conc}


	\clearpage{\thispagestyle{empty}\cleardoublepage}
	\chapter{Testes e Resultados}
\label{ch::testes}

\section{Introdução}
\label{sec::testes:intro}

Bla bla inicial\ldots


\section{Secções?}
A analisar\ldots


\begin{figure}[!htbp]
	\centering
	\includegraphics[width=.8\textwidth]{home}
	\caption[Ecrã inicial da aplicação]{Ecrã inicial da aplicação \theapp.}
	\label{fig::home}
\end{figure}

\begin{figure}[!htbp]
	\centering
	\includegraphics[width=.8\textwidth]{spheresphere}
	\caption[Teste do algoritmo de \textit{sphere tracing}]{Teste do algoritmo de \textit{sphere tracing} no \textit{website} \url{shadertoy.com} de uma esfera e um plano.}.
	\label{fig::spheresphere}
\end{figure}

\begin{figure}[!htbp]
	\centering
	\includegraphics[width=.8\textwidth]{ashalgosphere}
	\caption[Teste do algoritmo naïve]{Teste do algoritmo naïve no \textit{website} \url{shadertoy.com} de uma esfera.}
	\label{fig::ashalgosphere}
\end{figure}

\begin{figure}[!htbp]
	\centering
	\includegraphics[width=.8\textwidth]{sphereoriginal}
	\caption[Nove objetos com \textit{sphere tracing} no \theapp]{Renderização de nove objetos no \theapp~usando o algoritmo de \textit{sphere tracing}.}
	\label{fig::sphereoriginal}
\end{figure}

\begin{figure}[!htbp]
	\centering
	\includegraphics[width=.8\textwidth]{spheresmooth}
	\caption[Nove objetos com \textit{sphere tracing} e suavização no \theapp]{Renderização de nove objetos no \theapp~usando o algoritmo de \textit{sphere tracing} com um fator de suavização $s$ dependente do tempo de execução $t$, em particular $s = \cos(t)$.}
	\label{fig::spheresmooth}
\end{figure}

\begin{figure}[!htbp]
	\centering
	\includegraphics[width=.8\textwidth]{calcglsphere}
	\caption[Esfera no \theapp~com algoritmo naïve]{Esfera renderizada no \theapp, em fase inicial de testes, com o algoritmo naïve.}
	\label{fig::calcglsphere}
\end{figure}

\begin{figure}[!htbp]
	\centering
	\includegraphics[width=.8\textwidth]{calcglpisurf}
	\caption[Superfície $\Pi$ no \theapp~com algoritmo naïve]{Superfície $\Pi$ renderizada no \theapp~com algoritmo naïve.}
	\label{fig::calcglpisurf}
\end{figure}

\begin{figure}[!htbp]
	\centering
	\includegraphics[width=.8\textwidth]{calcglgenus}
	\caption[\textit{Genus} no \theapp~com algoritmo naïve]{\textit{Genus} renderizado no \theapp~com algoritmo naïve.}
	\label{fig::calcglgenus}
\end{figure}


\section{Conclusões}
\label{sec::testes:conc}

\ldots Whiskas Saquetas.

	\clearpage{\thispagestyle{empty}\cleardoublepage}
	\chapter{Conclusões e Trabalho Futuro}
\label{ch::conc}

\section{Conclusões}
\label{sec::conc:conc}

\todo{}


\section{Trabalho Futuro}
\label{sec::conc:futuro}

\todo{}

	\clearpage{\thispagestyle{empty}\cleardoublepage}
	
	% SE EXISTIREM APENDICES, DESCOMENTAR O QUE ESTÁ EM BAIXO
	% \appendix
	% \include{apendice1}
	% \clearpage{\pagestyle{empty}\cleardoublepage}
	% \include{continuacao}
	% \clearpage{\pagestyle{empty}\cleardoublepage}
	% \include{apendice2}
	% \clearpage{\pagestyle{empty}\cleardoublepage}
	% \include{apendice3}
	% \clearpage{\pagestyle{empty}\cleardoublepage}
	
	\backmatter
	
	\bibliographystyle{IEEEtran}
	\bibliography{bibliografia.bib}
\end{document}