%% Escolher conforme o relatório seja em português ou em inglês:
% \usepackage[english]{babel}
\usepackage[portuguese]{babel}

%% Pacotes de enconding:
% \usepackage[utf8]{inputenc}
\usepackage[T1]{fontenc}

%% Pacotes de base:
\usepackage{makeidx}
\usepackage{xspace}
\usepackage{graphicx,color,times}
\usepackage{fancyhdr}
\usepackage{minted}
\usepackage{paralist}
\usepackage{enumitem}
\usepackage{booktabs}
\usepackage{amssymb}
\usepackage{calc}
\usepackage{array}
\usepackage{xcolor}
\usepackage{caption}
\usepackage{subcaption}
\usepackage{amsmath}
\usepackage{latexsym}
\usepackage[printonlyused]{acronym}
\usepackage{float}
\usepackage{tocbibind}
\usepackage{algorithm}
\usepackage{algpseudocode}

%% Outros pacotes para usos menos comuns:
\usepackage{multirow}
% \usepackage{listings}
% \usepackage{pxfonts}
% \usepackage{times}
% \usepackage{mathptm}
% \usepackage{amssymb}
% \usepackage{amsfonts}
% \usepackage{color}
% \usepackage{colortbl}
% \usepackage{natbib}

%% Caso seja necessário um glossário, usar este pacote:
% \usepackage{glossaries}
% \makeglossaries

%% ATENÇÃO!!! Não mudar a ordem deste pacote face aos outros: vai dar cabo dos links no PDF final!
\usepackage{hyperref}

%% Muda a fonte de texto monospaced para algo ranhoso.
%% Não recomendo usar, mas fica aqui a opção dada pelo Inácio xD
% \renewcommand{\ttdefault}{phv}

%% --------------------------------------------------------------------------------
%% Formatação e personalização: não mexer se não souberes o que estás a fazer!
\pagestyle{fancy}
\renewcommand{\chaptermark}[1]{\markboth{#1}{}}
\renewcommand{\sectionmark}[1]{\markright{\thesection\ #1}}
\fancyhf{} \fancyhead[LE,RO]{\bfseries\thepage}
\fancyhead[LO]{\bfseries\rightmark}
\fancyhead[RE]{\bfseries\leftmark}
\renewcommand{\headrulewidth}{0.5pt}
\renewcommand{\footrulewidth}{0pt}
\setlength{\headheight}{15.6pt}
\setlength{\marginparsep}{0cm}
\setlength{\marginparwidth}{0cm}
\setlength{\marginparpush}{0cm}
\addtolength{\hoffset}{-1.0cm}
\addtolength{\oddsidemargin}{\evensidemargin}
\addtolength{\oddsidemargin}{0.5cm}
\addtolength{\evensidemargin}{-0.5cm}

\usepackage{fix-cm}
\usepackage{fourier}
\usepackage[scaled=.92]{helvet}
\definecolor{ChapGrey}{rgb}{0.6,0.6,0.6}
\newcommand{\LargeFont}{
	\usefont{\encodingdefault}{\rmdefault}{b}{n}
	\fontsize{60}{80}\selectfont\color{ChapGrey}
}
\makeatletter
\makechapterstyle{GreyNum}{
	\renewcommand{\chapnamefont}{\large\sffamily\bfseries\itshape}
	\renewcommand{\chapnumfont}{\LargeFont}
	\renewcommand{\chaptitlefont}{\Huge\sffamily\bfseries\itshape}
	\setlength{\beforechapskip}{0pt}
	\setlength{\midchapskip}{40pt}
	\setlength{\afterchapskip}{60pt}
	\renewcommand\chapterheadstart{\vspace*{\beforechapskip}}
	\renewcommand\printchaptername{
		\begin{tabular}{@{}c@{}}
			\chapnamefont \@chapapp\\}
		\renewcommand\chapternamenum{\noalign{\vskip 2ex}}
		\renewcommand\printchapternum{\chapnumfont\thechapter\par}
		\renewcommand\afterchapternum{
		\end{tabular}
		\par\nobreak\vskip\midchapskip}
	\renewcommand\printchapternonum{}
	\renewcommand\printchaptertitle[1]{
		{\chaptitlefont{##1}\par}}
	\renewcommand\afterchaptertitle{\par\nobreak\vskip \afterchapskip}
}
\makeatother
\chapterstyle{GreyNum}

\setcounter{tocdepth}{3}
\setsecnumdepth{subsubsection}

\renewcommand{\ttdefault}{lmtt}
%% --------------------------------------------------------------------------------


%% --------------------------------------------------------------------------------
%% ----- Cenas do Inácio para o pacote "listings" -----
%% NEW COLORS
%\definecolor{dark}{gray}{0.25}
%\definecolor{lgray}{gray}{0.9}
%\definecolor{dkblue}{rgb}{0,0.13,0.4}
%\definecolor{dkgreen}{rgb}{0,0.6,0}
%\definecolor{gray}{rgb}{0.5,0.5,0.5}
%\definecolor{mauve}{rgb}{0.58,0,0.82}

%\lstset{ %
%	language=C,                    basicstyle=\footnotesize,
%	numbers=none,                  numberstyle=\tiny\color{gray}, 
%	stepnumber=1,                  numbersep=5pt,
%	backgroundcolor=\color{white}, showspaces=false,
%	showstringspaces=false,        showtabs=false,
%	frame=single,                  rulecolor=\color{black},
%	tabsize=2,                     captionpos=b,
%	breaklines=true,               breakatwhitespace=false,
%	title=\lstname,                keywordstyle=\color{blue},
%	commentstyle=\color{dkgreen},  stringstyle=\color{mauve},
%	escapeinside={\%*}{*)},        morekeywords={*},
%	belowskip=0cm
%}
%
% \renewcommand{\lstlistingname}{Excerto de Código}
% \renewcommand{\lstlistlistingname}{Lista de Excertos de Código}
%
% \renewcommand{\today}{\day \ifcase \month \or janeiro\or fevereiro\or março\or %
	% abril\or maio\or junho\or julho\or agosto\or setembro\or outubro\or novembro\or %
	% dezembro\fi de \number \year}
%% --------------------------------------------------------------------------------


%% --------------------------------------------------------------------------------
%% ----- Outras personalizações próprias -----

%% Quote famosa
\newcommand{\famousquote}[2]{
	\begin{quote}
		\rule{\textwidth-2\leftmargin}{0.4pt}
		{\itshape #1}
		\vspace{-12pt}
		\begin{flushright}
			\textasciitilde~#2
		\end{flushright}
		\vspace{-20pt}
		\rule{\textwidth-2\leftmargin}{0.4pt}
	\end{quote}
}

%% Definições sobre figuras
\graphicspath{{./img/}}
\newcommand{\figuredefaultscale}{0.5}

%% Símbolos registered e copyright em superscript
\newcommand{\registered}{\textsuperscript{\textregistered}}
\newcommand{\ccopyright}{\textsuperscript{\textcopyright}}

%% Placeholders
\newcommand{\revision}[1]{{\color{orange}[!Rev] #1}}
\newcommand{\todo}[1]{{\color{red}[!TODO] #1}}
\newcommand{\hint}[1]{{\color{green!50!black}[!Hint] #1}}

%% Pacote "algorithm": ambiente com nome português
\floatname{algorithm}{Algoritmo}